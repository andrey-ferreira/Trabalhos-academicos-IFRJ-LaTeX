%
% Documento: Introdução
%
\chapter{Introdução}
\label{chap:introducao}

A classe \texttt{ifrjtex} consiste em uma extensão da classe \texttt{abn\TeX} e foi elaborada com o objetivo de facilitar a produção dos trabalhos acadêmicos do Instituto Federal de Educação, Ciência e Tecnologia do Rio de Janeiro por parte dos discentes e servidores da instituição, tratando-se de um projeto independente e não oficial.

Caso localize algum erro ou tenha sugestões de modificações, registre suas observações no \href{https://github.com/andrey-ferreira/ifrjTeX}{\textcolor{blue}{diretório do projeto}} ou entre em contato com o responsável pelo mesmo através do e-mail \href{mailto:andrey.ferreira@ifrj.edu.br}{\textcolor{blue}{andrey.ferreira@ifrj.edu.br}}.

Por se tratar de uma extensão {\ttfamily abntex2},  a constituição do documento torna-se facilitada, uma vez que o mesmo possui comandos especiais para auxiliar a distribuição/definição das diversas partes constituintes do projeto.
Esse estilo é baseado nas normas da ABNT\index{ABNT}.
Maiores detalhes relacionados aos comandos existentes no estilo poderão ser adquiridos através da documentação disponível no site \href{https://code.google.com/p/abntex2/}{https://code.google.com/p/abntex2/} \cite{abntex2classe}.

Uma das principais vantagens do uso do estilo de formatação para \LaTeX é a formatação \textit{automática} dos elementos que compõem um documento acadêmico, tais como capa, folha de rosto, dedicatória, agradecimentos, epígrafe, resumo, abstract, listas de figuras, tabelas, siglas e símbolos, sumário, capítulos, referências, etc.

Neste documento o leitor encontrará instruções específicas para o uso da classe \texttt{ifrjtex} e poucas instruções sobre a escrita de alguns elementos textuais em \LaTeX. 

Para melhor entendimento do uso do estilo de formatação, aconselha-se que o potencial usuário analise os comandos existentes no arquivo {\ttfamily main.tex} e os resultados obtidos no arquivo {\ttfamily main.pdf} depois do processamento pelo software LATEX + BIBTEX \cite{LaTeX2009,BibTeX2009}.
Recomenda-se a consulta ao material de referência do software para a sua correta utilização \cite{Lamport1986,Buerger1989,Kopka2003,Mittelbach2004}.